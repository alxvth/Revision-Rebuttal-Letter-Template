% ################################
%          Reviewer 1
% ################################
\subsection{Reviewer 1} \label{sec:rev:1}

This paper presents a method.

\subsubsection*{Strengths:}

\begin{itemize}[itemsep=1pt,parsep=1pt]
    \item[S1:] Useful
    \item[S2:] Useful
    \item[S3:] Useful
\end{itemize}

\noindent The source code is included. and available on GitHub.


\subsubsection*{Weaknesses:}

The mathematical is lacking in precision, correctness and consistency.

\begin{itemize}[itemsep=1pt,parsep=1pt]
    \item[W1:] Bad
    \item[W2:] Bad
    \item[W3:] Bad
\end{itemize}

\begin{response} \label{res:rev1:weak:math}
We revised our mathematical notation.
\end{response}


\subsubsection*{Detailed remarks:}

In section 3.2 three types of neighbourhood are introduced.

\begin{response} \label{res:rev1:detail:types}
Typos have been corrected.
\end{response}


\subsubsection*{Overall Rating}
\textbf{2 - Reject} \\
The paper is not ready for publication.

\paragraph{Justification}
Bla bla.

\paragraph{Expertise}
Knowledgeable

\paragraph{Confidence}
Very confident

\paragraph{Supplemental Materials}
Acceptable
